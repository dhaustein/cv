\documentclass[10pt, a4paper]{article}

% Packages:
\usepackage[
    ignoreheadfoot, % set margins without considering header and footer
    top=2 cm, % seperation between body and page edge from the top
    bottom=2 cm, % seperation between body and page edge from the bottom
    left=2 cm, % seperation between body and page edge from the left
    right=2 cm, % seperation between body and page edge from the right
    footskip=1.0 cm, % seperation between body and footer
    % showframe % for debugging
]{geometry} % for adjusting page geometry
\usepackage{titlesec} % for customizing section titles
\usepackage{tabularx} % for making tables with fixed width columns
\usepackage{array} % tabularx requires this
\usepackage[dvipsnames]{xcolor} % for coloring text
\definecolor{primaryColor}{RGB}{0, 0, 0} % define primary color
\definecolor{linkColor}{RGB}{0, 51, 180} % define dark blue color for links
\usepackage{enumitem} % for customizing lists
\usepackage{fontawesome5} % for using icons
\usepackage{amsmath} % for math
\usepackage[
    pdftitle={Dusan Haustein's CV},
    pdfauthor={Dusan Haustein},
    pdfcreator={LaTeX},
    colorlinks=true,
    urlcolor=linkColor
]{hyperref} % for links, metadata and bookmarks
\usepackage[pscoord]{eso-pic} % for floating text on the page
\usepackage{calc} % for calculating lengths
\usepackage{bookmark} % for bookmarks
\usepackage{lastpage} % for getting the total number of pages
\usepackage{changepage} % for one column entries (adjustwidth environment)
\usepackage{paracol} % for two and three column entries
\usepackage{ifthen} % for conditional statements
\usepackage{needspace} % for avoiding page brake right after the section title
\usepackage{iftex} % check if engine is pdflatex, xetex or luatex
\usepackage{ragged2e} % for justifying text

% Ensure that generate pdf is machine readable/ATS parsable:
\ifPDFTeX
    \input{glyphtounicode}
    \pdfgentounicode=1
    \usepackage[T1]{fontenc}
    \usepackage[utf8]{inputenc}
    \usepackage{lmodern}
\fi

\usepackage{charter}

% Some settings:
\raggedright
\AtBeginEnvironment{adjustwidth}{\partopsep0pt} % remove space before adjustwidth environment
\pagestyle{empty} % no header or footer
\setcounter{secnumdepth}{0} % no section numbering
\setlength{\parindent}{0pt} % no indentation
\setlength{\topskip}{0pt} % no top skip
\setlength{\columnsep}{0.15cm} % set column seperation
\pagenumbering{gobble} % no page numbering

\titleformat{\section}{\needspace{4\baselineskip}\bfseries\large}{}{0pt}{}[\vspace{1pt}\titlerule]

\titlespacing{\section}{
    % left space:
    -1pt
}{
    % top space:
    0.3 cm
}{
    % bottom space:
    0.2 cm
} % section title spacing

\renewcommand\labelitemi{$\vcenter{\hbox{\small$\bullet$}}$} % custom bullet points
\newenvironment{highlights}{
    \begin{itemize}[
        topsep=0.05 cm,
        parsep=0.05 cm,
        partopsep=0pt,
        itemsep=0pt,
        leftmargin=0 cm + 10pt
    ]
}{
    \end{itemize}
} % new environment for highlights


\newenvironment{highlightsforbulletentries}{
    \begin{itemize}[
        topsep=0.10 cm,
        parsep=0.10 cm,
        partopsep=0pt,
        itemsep=0pt,
        leftmargin=10pt
    ]
}{
    \end{itemize}
} % new environment for highlights for bullet entries

\newenvironment{onecolentry}{
    \begin{adjustwidth}{
        0 cm + 0.00001 cm
    }{
        0 cm + 0.00001 cm
    }
}{
    \end{adjustwidth}
} % new environment for one column entries

\newenvironment{twocolentry}[2][]{
    \onecolentry
    \def\secondColumn{#2}
    \setcolumnwidth{\fill, 4.5 cm}
    \begin{paracol}{2}
}{
    \switchcolumn \raggedleft \secondColumn
    \end{paracol}
    \endonecolentry
} % new environment for two column entries

\newenvironment{threecolentry}[3][]{
    \onecolentry
    \def\thirdColumn{#3}
    \setcolumnwidth{, \fill, 4.5 cm}
    \begin{paracol}{3}
    {\raggedright #2} \switchcolumn
}{
    \switchcolumn \raggedleft \thirdColumn
    \end{paracol}
    \endonecolentry
} % new environment for three column entries

\newenvironment{header}{
    \setlength{\topsep}{0pt}\par\kern\topsep\centering\linespread{1.5}
}{
    \par\kern\topsep
} % new environment for the header

\newcommand{\placelastupdatedtext}{% \placetextbox{<horizontal pos>}{<vertical pos>}{<stuff>}
  \AddToShipoutPictureFG*{% Add <stuff> to current page foreground
    \put(
        \LenToUnit{\paperwidth-2 cm-0 cm+0.05cm},
        \LenToUnit{\paperheight-1.0 cm}
    ){\vtop{{\null}\makebox[0pt][c]{
        \small\color{gray}\textit{Last updated in August 2025}\hspace{\widthof{Last updated in August 2025}}
    }}}%
  }%
}%

% save the original href command in a new command:
\let\hrefWithoutArrow\href

% new command for external links:
\newcommand{\AND}{$|$}
\newsavebox\ANDbox
\sbox\ANDbox{$|$}

\begin{document}
    \begin{header}
        \fontsize{25 pt}{25 pt}\selectfont Dusan Haustein

        \vspace{5 pt}

        \normalsize
        \mbox{Czech Republic}%
        \kern 5.0 pt%

        % \AND%
        % \kern 5.0 pt%
        \mbox{\hrefWithoutArrow{https://www.linkedin.com/in/dusanhaustein/}{linkedin.com/in/dusanhaustein/}}
        \hspace{35.0 pt}%
        \mbox{\hrefWithoutArrow{https://github.com/dhaustein}{github.com/dhaustein}}%
        \kern 5.0 pt%

    \end{header}

    \vspace{5 pt - 0.3 cm}


    \section{Summary}

        \begin{onecolentry}
            \justifying\hyphenpenalty=10\parindent=0pt
            Software quality engineer with 9 years of experience designing and delivering scalable, future-proof QA automation and testing processes.
            I specialize in QA team mentoring and leadership, test automation framework design, CI/CD implementation, and test strategy development.
            I have led distributed agile teams and worked on everything from greenfield startup projects to large enterprise platforms.
            \par\vspace{\baselineskip}
            I bring overlap with cloud technologies, proficiency with containerization and Kubernetes, and hands-on experience leading remote teams of engineers.
            I've contributed automated testing suites for data intelligence systems, Red Hat Ansible, or embedded systems.
            \par\vspace{\baselineskip}
            For more details and references, please see my LinkedIn profile (linked above).
        \end{onecolentry}

        \vspace{0.2 cm}


    %There are 7 unique entry types: \textit{BulletEntry}, \textit{TextEntry}, \textit{EducationEntry}, \textit{ExperienceEntry}, \textit{NormalEntry}, \textit{PublicationEntry}, and \textit{OneLineEntry}.

    \section{Experience}

        % Latest
        \begin{twocolentry}{
            \small\mbox{June 2025 - September 2025}
        }
            \textbf{Principal QA Engineer}, Norhern.tech, Contract, Remote\end{twocolentry}

        \vspace{0.10 cm}
        \begin{onecolentry}
            \begin{highlights}
                \item Develop and maintain automated test strategies for a client's IoT device management platform using Python.
                \item Create comprehensive test cases ensuring functionality, performance, scalability, and security of embedded software and SaaS components.
                \item Build and maintain CI/CD pipelines using GitHub/GitLab CI, Docker, and Kubernetes.
                \item Work with containerized environments and orchestration tools to support testing of distributed IoT systems.
                \item Contribute to continuous improvement of testing processes, methodologies, and software release workflows.
                \item Enable developers to run tests more efficiently through improved tooling and well-documented testing infrastructure.
                \item Tech stack: Python, pytest, Docker, Kubernetes, GitHub CI, GitLab CI, Linux, Go, C/C++
            \end{highlights}
        \end{onecolentry}

        \vspace{0.5 cm}

        % Red Hat
        \begin{twocolentry}{
            \small\mbox{November 2023 - May 2025}
        }
            \textbf{Senior Software Quality Engineer}, Red Hat, Full-time, Remote\end{twocolentry}

        \vspace{0.10 cm}

        \begin{onecolentry}
            \begin{highlights}
                \item Worked on the test automation suite and CI/CD pipeline for the Event-driven component of Ansible Automation Platform.
                \item Helped implement a platform-wide test suite for use by other engineers.
                \item Collaborated with the development team and business unit on expanding test coverage for new features.
                \item Led the testing effort for various features released as part of the Ansible platform, such as RBAC.
                \item Tech stack: Python, Ansible, Linux, Jenkins, Podman
            \end{highlights}
        \end{onecolentry}

        \vspace{0.5 cm}

        % Manta
        \begin{twocolentry}{
            \small\mbox{January 2023 - October 2023}
        }
            \textbf{Test Automation Lead}, MANTA, Full-time, Hybrid\end{twocolentry}

        \vspace{0.10 cm}
        \begin{onecolentry}
            \begin{highlights}
                \item Led a distributed agile team of test automation engineers responsible for implementing and maintaining an in-house automated testing framework built on top of Robot Framework and Python & pytest.
                \item Collaborated with other members of the engineering department to define and implement testing best practices and standards.
                \item Worked with stakeholders to define and prioritize testing requirements for new features and releases, ensuring adequate testing and quality standards.
                \item Assessed the automation needs of other development teams and identified areas where automation could be most beneficial.
                \item Owned and maintained the testing Jenkins and Azure CI/CD pipelines.
                \item Monitored the testing environment, analyzed test data, and reported on test results.
            \end{highlights}
        \end{onecolentry}

        \vspace{0.5 cm}

        % Manta
        \begin{twocolentry}{
            \small\mbox{October 2021 - January 2023}
        }
            \textbf{Senior Test Automation Engineer}, MANTA, Full-time, Hybrid\end{twocolentry}

        \vspace{0.10 cm}
        \begin{onecolentry}
            \begin{highlights}
                \item Created a greenfield test automation pipeline for the product using Robot Framework.
                \item Analyzed and developed test automation strategy for a complex, multi-component enterprise product consisting of a React-based web application, Java back-end, REST APIs, and CLI components.
                \item Designed automated test cases using Robot Framework and Python.
                \item Executed functional, regression, performance, and acceptance automated tests as part of the CI cycle.
                \item Automated security tests with OWASP ZAP.
                \item Designed and maintained the testing component of the CI/CD pipeline.
                \item Assisted in the migration of the application to the cloud.
                \item Coordinated with the development and customer success teams, and other stakeholders to deliver the best possible test automation strategy.
                \item Tech stack: Robot Framework, Jenkins CI, Python, Ansible, AWS EC2, Vagrant, Docker, jMeter, ZAP, k6
            \end{highlights}
        \end{onecolentry}

        \vspace{0.5 cm}

        % PeoplePath
        \begin{twocolentry}{
            \small\mbox{October 2020 - September 2021}
        }
            \textbf{Business Analyst}, PeoplePath, Full-time, Hybrid\end{twocolentry}

        \vspace{0.10 cm}
        \begin{onecolentry}
            \begin{highlights}
                \item Translated new business requirements into clearly defined, achievable tasks for the development team.
                \item Managed and prioritized the project backlog and processed feedback from stakeholders.
                \item Maintained the product's technical documentation and assisted the customer success team with configuration.
                \item Drove future product development initiatives.
            \end{highlights}
        \end{onecolentry}

        \vspace{0.5 cm}

        % PeoplePath
        \begin{twocolentry}{
            \small\mbox{September 2016 - September 2020}
        }
            \textbf{QA Engineer}, PeoplePath, Full-time, On-site\end{twocolentry}

        \vspace{0.10 cm}
        \begin{onecolentry}
            \begin{highlights}
                \item Worked closely with product management and software analysts to understand business requirements and technical specifications.
                \item Enhanced product quality by automating functional and regression tests.
                \item Collaborated with the support team on production defects and provided root cause analysis.
                \item Supported the development team with knowledge of the product's business logic.
                \item Tech stack: Jenkins CI, Docker, PHP, Behat, Cucumber/Gherkin, Selenium, Python
            \end{highlights}
        \end{onecolentry}

    \vspace{0.2 cm}

    \section{Education}

        % Master
        \begin{twocolentry}{
            \small2013 - 2015
        }
            \textbf{University of West Bohemia}, Master's degree, Cognitive sciences\end{twocolentry}

        \vspace{0.10 cm}

        % Bachelor
        \begin{twocolentry}{
            \small2009 - 2013
        }
            \textbf{University of West Bohemia}, Bachelor's degree, Linguistics\end{twocolentry}

    \vspace{0.2 cm}

    \section{Courses \& Certifications}

        \begin{twocolentry}{
            \small Jan 2025
        }
            \textbf{Various Django and Python development courses}, see \hrefWithoutArrow{https://www.linkedin.com/in/dusanhaustein/details/certifications/}{LinkedIn profile for details}
        \end{twocolentry}

        \vspace{0.10 cm}

        \begin{twocolentry}{
            \small 2019
        }
            \textbf{AST - Black Box Software Testing Foundations Certificate}
        \end{twocolentry}

        \vspace{0.10 cm}

        \begin{twocolentry}{
            \small2015
        }
            \textbf{Cambridge - Certificate of Proficiency in English (CPE) Level C2}
        \end{twocolentry}

        \vspace{0.2 cm}

    \section{Technologies}


        \begin{onecolentry}
            \textbf{Languages:} Python, Bash, Groovy, PHP, TypeScript
        \end{onecolentry}

        \vspace{0.2 cm}

        \begin{onecolentry}
            \textbf{Technologies:} Ansible, Robot Framework, Selenium, Jenkins, Gitlab CI, Kubernetes, Docker, AWS, QEMU, Linux, jMeter, ZAP, k6
        \end{onecolentry}


\end{document}
