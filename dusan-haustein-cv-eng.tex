\documentclass[10pt, letterpaper]{article}

% Packages:
\usepackage[
    ignoreheadfoot, % set margins without considering header and footer
    top=2 cm, % seperation between body and page edge from the top
    bottom=2 cm, % seperation between body and page edge from the bottom
    left=2 cm, % seperation between body and page edge from the left
    right=2 cm, % seperation between body and page edge from the right
    footskip=1.0 cm, % seperation between body and footer
    % showframe % for debugging
]{geometry} % for adjusting page geometry
\usepackage{titlesec} % for customizing section titles
\usepackage{tabularx} % for making tables with fixed width columns
\usepackage{array} % tabularx requires this
\usepackage[dvipsnames]{xcolor} % for coloring text
\definecolor{primaryColor}{RGB}{0, 0, 0} % define primary color
\usepackage{enumitem} % for customizing lists
\usepackage{fontawesome5} % for using icons
\usepackage{amsmath} % for math
\usepackage[
    pdftitle={John Doe's CV},
    pdfauthor={John Doe},
    pdfcreator={LaTeX with RenderCV},
    colorlinks=true,
    urlcolor=primaryColor
]{hyperref} % for links, metadata and bookmarks
\usepackage[pscoord]{eso-pic} % for floating text on the page
\usepackage{calc} % for calculating lengths
\usepackage{bookmark} % for bookmarks
\usepackage{lastpage} % for getting the total number of pages
\usepackage{changepage} % for one column entries (adjustwidth environment)
\usepackage{paracol} % for two and three column entries
\usepackage{ifthen} % for conditional statements
\usepackage{needspace} % for avoiding page brake right after the section title
\usepackage{iftex} % check if engine is pdflatex, xetex or luatex

% Ensure that generate pdf is machine readable/ATS parsable:
\ifPDFTeX
    \input{glyphtounicode}
    \pdfgentounicode=1
    \usepackage[T1]{fontenc}
    \usepackage[utf8]{inputenc}
    \usepackage{lmodern}
\fi

\usepackage{charter}

% Some settings:
\raggedright
\AtBeginEnvironment{adjustwidth}{\partopsep0pt} % remove space before adjustwidth environment
\pagestyle{empty} % no header or footer
\setcounter{secnumdepth}{0} % no section numbering
\setlength{\parindent}{0pt} % no indentation
\setlength{\topskip}{0pt} % no top skip
\setlength{\columnsep}{0.15cm} % set column seperation
\pagenumbering{gobble} % no page numbering

\titleformat{\section}{\needspace{4\baselineskip}\bfseries\large}{}{0pt}{}[\vspace{1pt}\titlerule]

\titlespacing{\section}{
    % left space:
    -1pt
}{
    % top space:
    0.3 cm
}{
    % bottom space:
    0.2 cm
} % section title spacing

\renewcommand\labelitemi{$\vcenter{\hbox{\small$\bullet$}}$} % custom bullet points
\newenvironment{highlights}{
    \begin{itemize}[
        topsep=0.10 cm,
        parsep=0.10 cm,
        partopsep=0pt,
        itemsep=0pt,
        leftmargin=0 cm + 10pt
    ]
}{
    \end{itemize}
} % new environment for highlights


\newenvironment{highlightsforbulletentries}{
    \begin{itemize}[
        topsep=0.10 cm,
        parsep=0.10 cm,
        partopsep=0pt,
        itemsep=0pt,
        leftmargin=10pt
    ]
}{
    \end{itemize}
} % new environment for highlights for bullet entries

\newenvironment{onecolentry}{
    \begin{adjustwidth}{
        0 cm + 0.00001 cm
    }{
        0 cm + 0.00001 cm
    }
}{
    \end{adjustwidth}
} % new environment for one column entries

\newenvironment{twocolentry}[2][]{
    \onecolentry
    \def\secondColumn{#2}
    \setcolumnwidth{\fill, 4.5 cm}
    \begin{paracol}{2}
}{
    \switchcolumn \raggedleft \secondColumn
    \end{paracol}
    \endonecolentry
} % new environment for two column entries

\newenvironment{threecolentry}[3][]{
    \onecolentry
    \def\thirdColumn{#3}
    \setcolumnwidth{, \fill, 4.5 cm}
    \begin{paracol}{3}
    {\raggedright #2} \switchcolumn
}{
    \switchcolumn \raggedleft \thirdColumn
    \end{paracol}
    \endonecolentry
} % new environment for three column entries

\newenvironment{header}{
    \setlength{\topsep}{0pt}\par\kern\topsep\centering\linespread{1.5}
}{
    \par\kern\topsep
} % new environment for the header

\newcommand{\placelastupdatedtext}{% \placetextbox{<horizontal pos>}{<vertical pos>}{<stuff>}
  \AddToShipoutPictureFG*{% Add <stuff> to current page foreground
    \put(
        \LenToUnit{\paperwidth-2 cm-0 cm+0.05cm},
        \LenToUnit{\paperheight-1.0 cm}
    ){\vtop{{\null}\makebox[0pt][c]{
        \small\color{gray}\textit{Last updated in September 2024}\hspace{\widthof{Last updated in August 2025}}
    }}}%
  }%
}%

% save the original href command in a new command:
\let\hrefWithoutArrow\href

% new command for external links:


\begin{document}
    \newcommand{\AND}{\unskip
        \cleaders\copy\ANDbox\hskip\wd\ANDbox
        \ignorespaces
    }
    \newsavebox\ANDbox
    \sbox\ANDbox{$|$}

    \begin{header}
        \fontsize{25 pt}{25 pt}\selectfont Dusan Haustein

        \vspace{5 pt}

        \normalsize
        \mbox{Czech Republic}%
        \kern 5.0 pt%

        \AND%
        \kern 5.0 pt%
        \mbox{\hrefWithoutArrow{https://www.linkedin.com/in/dusanhaustein/}{linkedin.com/in/dusanhaustein/}}%
        \kern 5.0 pt%

        \AND%
        \kern 5.0 pt%
        \mbox{\hrefWithoutArrow{https://github.com/dhaustein}{github.com/dhaustein}}%

    \end{header}

    \vspace{5 pt - 0.3 cm}


    \section{Summary}

        \begin{onecolentry}
            Software quality engineer of 9 years of experience with testing, test automation and CI/CD implementation.

            I worked for teams small and large. I have experience with managing a remote team of test automation engineers (5 people) as well.

            I am most skilled with Python, Linux, virtualization, containerization, CI/CD, test automation, Kubernetes, and I like mentoring others.

            For more details and references please see my LinkedIn profile (linked above).
        \end{onecolentry}

        \vspace{0.2 cm}


    %There are 7 unique entry types: \textit{BulletEntry}, \textit{TextEntry}, \textit{EducationEntry}, \textit{ExperienceEntry}, \textit{NormalEntry}, \textit{PublicationEntry}, and \textit{OneLineEntry}.

    \section{Experience}

        % Latest
        \begin{twocolentry}{
            June 2025 - Present
        }
            \textbf{Principal QA Engineer}, Freelance, Remote\end{twocolentry}

        \vspace{0.10 cm}
        \begin{onecolentry}
            \begin{highlights}
                \item Develop and maintain automated test strategies for a client's IoT device management platform using Python.
                \item Create comprehensive test cases ensuring functionality, performance, scalability, and security of embedded software and SaaS components.
                \item Build and maintain CI/CD pipelines using GitHub/GitLab CI, Docker, and Kubernetes.
                \item Work with containerized environments and orchestration tools to support testing of distributed IoT systems.
                \item Contribute to continuous improvement of testing processes, methodologies, and software release workflows.
                \item Enable developers to run tests more efficiently through improved tooling and well-documented testing infrastructure.
                \item Tech stack: Python, pytest, Docker, Kubernetes, GitHub CI, GitLab CI, Linux, Go, C/C++
            \end{highlights}
        \end{onecolentry}

        \vspace{0.2 cm}

        % Red Hat
        \begin{twocolentry}{
            November 2023 - May 2025
        }
            \textbf{Senior Software Quality Engineer}, Red Hat, Full-time, Remote\end{twocolentry}

        \vspace{0.10 cm}

        \begin{onecolentry}
            \begin{highlights}
                \item Worked on the test automation suite and CI/CD pipeline for the Event-driven component of Ansible Automation Platform.
                \item Helped implement platform-wide test suite to be used by other engineers.
                \item Collaborated with the development team and business unit on expanding test coverage over new features.
                \item Lead the testing effort for various features released as part of the Ansible platform, such as RBAC.
                \item Tech stack: Python, Ansible, Linux, Jenkins, Podman
            \end{highlights}
        \end{onecolentry}

        \vspace{0.2 cm}

        % Manta
        \begin{twocolentry}{
            January 2023 - October 2023
        }
            \textbf{Test Automation Lead}, MANTA, Full-time, Hybrid\end{twocolentry}

        \vspace{0.10 cm}
        \begin{onecolentry}
            \begin{highlights}
                \item Lead a distributed agile team of test automation engineers responsible for implementing and maintaining an in-house automated testing framework built atop of Robot Framework and Playwright.
                \item Collaborate with other members of the engineering department to define and implement testing best practices and standards.
                \item Work with stakeholders to define and prioritize testing requirements for new features and releases, ensuring that they are adequately tested and meet the expected quality standard.
                \item Assess the automation needs of other development teams and identify areas where automation can be most helpful to them.
                \item Own and maintain the testing Jenkins and Azure CI/CD pipelines.
                \item Monitor the testing environment, analyze test data, and report on test results.
            \end{highlights}
        \end{onecolentry}

        \vspace{0.2 cm}

        % Manta
        \begin{twocolentry}{
            October 2021 - January 2023
        }
            \textbf{Senior Test Automation Engineer}, MANTA, Full-time, Hybrid\end{twocolentry}

        \vspace{0.10 cm}
        \begin{onecolentry}
            \begin{highlights}
                \item Created a greenfield test automation pipeline for the product using Robot Framework.
                \item Analyze and develop test automation strategy for a complex, multi-component enterprise product consisting of a React based web application, Java back-end, REST APIs, and CLI components.
                \item Designed automation test cases using the Robot Framework and Python.
                \item Executed functional, regression, performance and acceptance automated tests as part of the CI cycle.
                \item Automated security tests with OWASP ZAP.
                \item Designed and maintain the testing part of CI/CD pipeline.
                \item Helped in the migration of the application to the cloud.
                \item Coordinated with the development and customer success team, and other stakeholders to bring the best possible test automation strategy.
                \item Tech stack: Robot Framework, Jenkins CI, Playwright, Python, Ansible, AWS EC2, Vagrant, Docker, git
            \end{highlights}
        \end{onecolentry}

        \vspace{0.2 cm}

        % PeoplePath
        \begin{twocolentry}{
            October 2020 - September 2021
        }
            \textbf{Business Analyst}, PeoplePath, Full-time, Hybrid\end{twocolentry}

        \vspace{0.10 cm}
        \begin{onecolentry}
            \begin{highlights}
                \item Translated new business requirements into clearly defined, achievable tasks for the development team.
                \item Managed and prioritize projects backlog and process feedback from stakeholders.
                \item Maintained product's technical documentation and help the customer success team with configuration.
                \item Drove future product development.
            \end{highlights}
        \end{onecolentry}

        \vspace{0.2 cm}

        % PeoplePath
        \begin{twocolentry}{
            September 2016 - September 2020
        }
            \textbf{QA Engineer}, PeoplePath, Full-time, On-site\end{twocolentry}

        \vspace{0.10 cm}
        \begin{onecolentry}
            \begin{highlights}
                \item Worked closely with product management and software analysis to understand business requirements and technical specifications.
                \item Helped push our product's quality further by automating functional and regressions tests.
                \item Worked with the support team on possible production defects and provide root cause analysis.
                \item Provided support for the development team with my knowledge of product's business logic.
                \item Tech stack: Jenkins CI, Docker, PHP, Behat, Cucumber/Gherkin, Selenium, Python
            \end{highlights}
        \end{onecolentry}

    \section{Education}

        % Master
        \begin{twocolentry}{
            2013 - 2015
        }
            \textbf{University of West Bohemia}, Master's degree, Cognitive sciences\end{twocolentry}

        \vspace{0.10 cm}

        % Bachelor
        \begin{twocolentry}{
            2009 - 2013
        }
            \textbf{University of West Bohemia}, Bachelor's degree, Linguistics\end{twocolentry}

    \section{Courses & Certifications}

        \begin{samepage}
            \begin{twocolentry}{
                Jan 2025
            }
                \textbf{Various Django and Python development courses (see my LinkedIn)}
            \end{twocolentry}

            \vspace{0.10 cm}

            \begin{onecolentry}
                \begin{twocolentry}{
                February 2019
            }
                \textbf{AST - Black Box Software Testing Foundations certificate}

            \vspace{0.10 cm}

            \begin{twocolentry}{}
                \textbf{Cambridge - Certificate of Proficiency in English (CPE) Level C2}
            \end{twocolentry}

            \vspace{0.10 cm}

        \href{https://doi.org/10.1109/TASC.2023.3340648}{10.1109/TASC.2023.3340648}
        \end{onecolentry}

        \end{samepage}



        \section{Technologies}


        \begin{onecolentry}
            \textbf{Languages:} Python, Bash, Groovy, PHP, TypeScript
        \end{onecolentry}

        \vspace{0.2 cm}

        \begin{onecolentry}
            \textbf{Technologies:} Ansible, Robot Framework, Playwright, Selenium, Jenkins, Gitlab CI, Kubernetes, Docker, AWS, QEMU, Linux
        \end{onecolentry}


\end{document}
